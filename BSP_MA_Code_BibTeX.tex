%
% FH Technikum Wien
% !TEX encoding = UTF-8 Unicode
%
% Erstellung von Master- und Bachelorarbeiten an der FH Technikum Wien mit Hilfe von LaTeX und der Klasse TWBOOK
%
% Um ein eigenes Dokument zu erstellen, müssen Sie folgendes ergänzen:
% 1) Mit \documentclass[..] einstellen: Master- oder Bachelorarbeit, Studiengang und Sprache
% 2) Mit \newcommand{\FHTWCitationType}.. Zitierstandard festlegen (wird in der Regel vom Studiengang vorgegeben - bitte erfragen)
% 3) Deckblatt, Kurzfassung, etc. ausfüllen
% 4) und die Arbeit schreiben (die verwendeten Literaturquellen in Literatur.bib eintragen)
%
% Getestet mit TeXstudio mit Zeichenkodierung ISO-8859-1 (=ansinew/latin1) und MikTex unter Windows
% Zu beachten ist, dass die Kodierung der Datei mit der Kodierung des paketes inputenc zusammen passt!
% Die Kodierung der Datei twbook.cls MUSS ANSI betragen!
% Bei der Verwendung von UTF8 muss dnicht nur die Kodierung des Dokuments auf UTF8 gestellt sein, sondern auch die des BibTex-Files!
%
% Bugreports und Feedback bitte per E-Mail an latex@technikum-wien.at
%
% Versionen
% *) V0.7: 9.1.2015, RO: Modeline angepasst und verschoben
% *) V0.6: 10.10.2014, RO: Weitere Anpassung an die UK
% *) V0.5: 8.8.2014, WK: Literaturquellen überarbeitet und angepasst
% *) V0.4: 4.8.2014, WK: Initalversion in SVN eingespielt
%
\documentclass[Bachelor,BIF,english]{twbook}
\usepackage[utf8]{inputenc}
\usepackage[T1]{fontenc}

%
% Bitte in der folgenden Zeile den Zitierstandard festlegen
\newcommand{\FHTWCitationType}{IEEE} % IEEE oder HARVARD möglich - wenn Sie zwischen IEEE und HARVARD wechseln, bitte die temorären Dateien (aux, bbl, ...) löschen
%
\ifthenelse{\equal{\FHTWCitationType}{HARVARD}}{\usepackage{harvard}}{\usepackage{bibgerm}}

% Definition Code-Listings Formatierung:
\usepackage[final]{listings}
\lstset{captionpos=b, numberbychapter=false,caption=\lstname,frame=single, numbers=left, stepnumber=1, numbersep=2pt, xleftmargin=15pt, framexleftmargin=15pt, numberstyle=\tiny, tabsize=3, columns=fixed, basicstyle={\fontfamily{pcr}\selectfont\footnotesize}, keywordstyle=\bfseries, commentstyle={\color[gray]{0.33}\itshape}, stringstyle=\color[gray]{0.25}, breaklines, breakatwhitespace, breakautoindent}
\lstloadlanguages{[ANSI]C, C++, [gnu]make, gnuplot, Matlab}

%Formatieren des Quellcodeverzeichnisses
\makeatletter
% Setzen der Bezeichnungen für das Quellcodeverzeichnis/Abkürzungsverzeichnis in Abhängigkeit von der eingestellten Sprache
\providecommand\listacroname{}
\@ifclasswith{twbook}{english}
{%
    \renewcommand\lstlistingname{Code}
    \renewcommand\lstlistlistingname{List of Code}
    \renewcommand\listacroname{List of Abbreviations}
}{%
    \renewcommand\lstlistingname{Quellcode}
    \renewcommand\lstlistlistingname{Quellcodeverzeichnis}
    \renewcommand\listacroname{Abkürzungsverzeichnis}
}
% Wenn die Option listof=entryprefix gewählt wurde, Definition des Entyprefixes für das Quellcodeverzeichnis. Definition des Macros listoflolentryname analog zu listoflofentryname und listoflotentryname der KOMA-Klasse
\@ifclasswith{scrbook}{listof=entryprefix}
{%
    \newcommand\listoflolentryname\lstlistingname
}{%
}
\makeatother
\newcommand{\listofcode}{\phantomsection\lstlistoflistings}

% Die nachfolgenden Pakete stellen sonst nicht benötigte Features zur Verfügung
\usepackage{blindtext}
\usepackage{parskip}

%
% Einträge für Deckblatt, Kurzfassung, etc.
%
\title{A Comparative View of Cross-compiled and Interpreted Cross-Platform Approaches for iOS and Android Mobile Application Development}
\author{Dominik Hack}
\studentnumber{1610257044}
\supervisor{Dipl.-Ing. Dr.techn. Thomas Polzer}
\place{Vienna}
\kurzfassung{kurzfassung}
\schlagworte{Schlagwort1, Schlagwort2, Schlagwort3, Schlagwort4}
\outline{abstract}
\keywords{Keyword1, Keyword2, Keyword3, Keyword4}

\begin{document}

%Festlegungen für den HARVARD-Zitierstandard
\ifthenelse{\equal{\FHTWCitationType}{HARVARD}}{
\bibliographystyle{Harvard_FHTW_MR}%Zitierstandard FH Technikum Wien, Studiengang Mechatronik/Robotik, Version 1.2e
\citationstyle{dcu}%Correct citation-style (Harvardand, ";" between citations, "," between author and year)
\citationmode{abbr}%use "et al." with first citation
\iflanguage{ngerman}{
    %Deutsch Neue Rechtschreibung
    \newcommand{\citepic}[1]{(Quelle: \protect\cite{#1})}%Zitat: Bild
    \newcommand{\citefig}[2]{(Quelle: \protect\cite{#1}, S. #2)}%Zitat: Bild aus Dokument
    \newcommand{\citefigm}[2]{(Quelle: modifiziert "ubernommen aus \protect\cite{#1}, S. #2)}%Zitat: modifiziertes Bild aus Dokument
    \newcommand{\citep}{\citeasnoun}%In-Line Zitiat entweder mit \citep{} oder \citeasnoun{}
    \newcommand{\acessedthrough}{Verf{\"u}gbar unter:}%Für URL-Angabe
    \newcommand{\acessedthroughp}{Verf{\"u}gbar bei:}%Für URL-Angabe (Geschützte Datenbank, Zugriff durch FH)
    \newcommand{\acessedat}{Zugang am}%Für URL-Datum-Angabe
    \newcommand{\singlepage}{S.}%Für Seitenangabe (einzelne Seite)
    \newcommand{\multiplepages}{S.}%Für Seitenangabe (mehrere Seiten)
    \newcommand{\chapternr}{K.}%Für Kapitelangabe
    \renewcommand{\harvardand}{\&}%Harvardand in Zitaten
    \newcommand{\abstractonly}{ausschließlich Abstract}
    \newcommand{\edition}{. Auflage}%Angabe der Auflage
}{
\iflanguage{german}{
    %Deutsch
    \newcommand{\citepic}[1]{(Quelle: \protect\cite{#1})}%Zitat: Bild
    \newcommand{\citefig}[2]{(Quelle: \protect\cite{#1}, S. #2)}%Zitat: Bild aus Dokument
    \newcommand{\citefigm}[2]{(Quelle: modifiziert "ubernommen aus \protect\cite{#1}, S. #2)}%Zitat: modifiziertes Bild aus Dokument
    \newcommand{\citep}{\citeasnoun}%In-Line Zitiat entweder mit \citep{} oder \citeasnoun{}
    \newcommand{\acessedthrough}{Verf{\"u}gbar unter:}%Für URL-Angabe
    \newcommand{\acessedthroughp}{Verf{\"u}gbar bei:}%Für URL-Angabe (Geschützte Datenbank, Zugriff durch FH)
    \newcommand{\acessedat}{Zugang am}%Für URL-Datum-Angabe
    \newcommand{\singlepage}{S.}%Für Seitenangabe (einzelne Seite)
    \newcommand{\multiplepages}{S.}%Für Seitenangabe (mehrere Seiten)
    \newcommand{\chapternr}{K.}%Für Kapitelangabe
    \renewcommand{\harvardand}{\&}%Harvardand in Zitaten
    \newcommand{\abstractonly}{ausschließlich Abstract}
    \newcommand{\edition}{. Auflage}%Angabe der Auflage
}{
    %Englisch
    \newcommand{\citepic}[1]{(Source: \protect\cite{#1})}%Zitat: Bild
    \newcommand{\citefig}[2]{(Source: \protect\cite{#1}, p. #2)}%Zitat: Bild aus Dokument
    \newcommand{\citefigm}[2]{(Source: taken with modification from \protect\cite{#1}, p. #2)}%Zitat: modifiziertes Bild aus Dokument
    \newcommand{\citep}{\citeasnoun}%In-Line Zitiat entweder mit \citep{} oder \citeasnoun{}
    \newcommand{\acessedthrough}{Available at:}%Für URL-Angabe
    \newcommand{\acessedthroughp}{Available through:}%Für URL-Angabe (Geschützte Datenbank, Zugriff durch FH)
    \newcommand{\acessedat}{Accessed}%Für URL-Datum-Angabe
    \newcommand{\singlepage}{p.}%Für Seitenangabe (einzelne Seite)
    \newcommand{\multiplepages}{pp.}%Für Seitenangabe (mehrere Seiten)
    \newcommand{\chapternr}{Ch.}%Für Kapitelangabe
    \renewcommand{\harvardand}{\&}%Harvardand in Zitaten
    \newcommand{\abstractonly}{Abstract only}
    \newcommand{\edition}{~edition}%Edition -> note, that you have to write "edition = {2nd},"!
}}}

\maketitle
\chapter{Introduction}
% Motivation (warum arbeitet man überhaupt daran?, allgemeine Einführung in Themengebiete (Einstieg soll sich auf Wissen der Zielgruppe beschränken), State of the art zeigen (Nennung von bereits laufenden bzw. abgeschlossenen Forschungen), Forschungsfragen die man mittels dieser Arbeit beantworten will (hier beginnt roter Faden der Arbeit!), Überblick über die folgenden Kapitel
Motivation how mobile devices became more and more and how much there are right now; 
\\[\baselineskip]
general introduction regarding mobile development and cross platform; 
\\[\baselineskip]
Lots of studies on cross platform mobile development were already performed. Latif, Lakhrissi, Nfaoui and Es-Sbai published a paper \cite{7479278} in which they summarized all mobile cross platform development approaches and compared them with each other. As a result of their analysis they identified multiple desirable requirements of any cross platform technology. These requirements are: application scalability and maintainability, access to features of device, auto-optimization of resource consumption, security and a development environment that features intelligent auto-completion systems, debuggers, compilers and simulators for all supported platforms. Another paper \cite{7934674} published by the same authors also touches upon mobile cross platform development approaches but concentrates on comparing frameworks that use these approaches. In the following chapters of this thesis all mobile development approaches will further be explained and the benefits along with the disadvantages of each will be listed.
\\[\baselineskip]
React Native By implementing a React Native feature set and analyzing it through performance parameters \cite{JohanssonSderberg2018}, comparing applications developed in Xamarin and PhoneGap (firma!) with each other and native applications created in Android and iOS \cite{Armgren_2015}. In this thesis various noteworthy results of papers, studies and theses (\cite{Danielsson_2016}, \cite{Axelsson2016}, \cite{Hansson_Vidhall_2016}, \cite{MartinezLecomte2018}, \cite{ZubaBernhard2017EdPb}, \cite{WillocxVossaertNaessens2015}, \cite{MartinezLecomte2017}, \cite{Dickson_2013}, \cite{GaouarBenamarBendimerad2016}, \cite{7479278}, \cite{LinckArne2016}, \cite{7934674}) that either experiment with Xamarin or React Native or both will be summarized and their results presented. Besides performance this thesis will also include other criteria like code sharing, documentation and look and feel. If possible the results of each criteria will contain a discussion comparing Xamarin and React Native with each other. 
\\[\baselineskip]
The following are the research questions to be dealt with in this thesis:
\begin{itemize}
\item Does an application developed in Xamarin perform as well as one developed in React Native and which performs better compared to a native application created in Android or iOS?
\item How much code of an application written using Xamarin or React Native can be shared between Android and iOS?
\item Do the documentations of Xamarin and React Native lack information needed to develop features?
\item Will the look and feel of a native application created in Android or iOS stay the same when developed in Xamarin or React Native?
\end{itemize}

\chapter{Mobile Development Approaches}
To this day, lots of companies, if not all, recognized that offering mobile applications on smartphones or other mobile devices to their customers is crucial to have a chance against the competition \cite[p.~1]{7479278}. If a company chooses to provide a mobile application for their customers they do not want to offer the mobile application to only half of their customers. However to reach all customers a company has to develop their mobile application for multiple platforms \cite[p.~5]{Steczko2016}.
\\[\baselineskip]
Nevertheless, without precisely planning the development and maintenance of multiple mobile applications for various platforms the company could invest more money into the development and maintenance than possibly necessary \cite[p.~1]{JohanssonSderberg2018} \cite[p.~8]{Steczko2016} \cite[p.~757]{Ciman2014}. One of the things to consider is what framework should be used for the development of the application. There are lots of frameworks for mobile application development out there and nearly every framework was build by applying one mobile development approach. Each approach has different advantages and disadvantages, therefore it could be the best or worst depending on the situation \cite{6420693} \cite{7934674}. This means in order to minimize costs and thereby increasing the chance of the projects success it is essential to choose a framework with a suitable mobile development approach for the given situation. To assist in finding the most valuable approach for a project, the following points summarize the most used mobile development approaches. 

\section{Native Approach}
Choosing the native development approach implicates developing the same application separately for each platform it should be available on. Usually the most commonly used platforms on the market, which to this day are Android (insert company) and iOS (insert company), are chosen to be supported \cite[p.~5]{Steczko2016}. As a result, two applications have to be developed one for Android by using the Android-SDK and Java as the programming language, and one for iOS by using the iOS-SDK and Swift as the programming language \cite[p.~5]{LinckArne2016} \cite{AppleGettingStarted}. Even though this leads to the requirement of different skill sets and therefore potentially more developers being necessary for the development of the application, it also leads to applications developed by utilizing the official platform specific frameworks and tools recommended by the publishers of the platforms. 
\\[\baselineskip]
Through using platform specific frameworks and tools, a developer is able to directly communicate with the operating system of the mobile device to get access to features like sensors or cloud services without having to rely on other alternatives, that can potentially decrease performance, cross-platform applications have to use to communicate with the operating system and get access to the above mentioned features \cite[p.~6]{LinckArne2016}.
\\[\baselineskip]
On the downside each application has a different code base thus each code base has to be maintained. This either means that one developer with knowledge of developing on all platforms needs to work on maintenance for all applications or multiple developers with special knowledge regarding one platform work on the maintenance of each particular platform's application \cite[p.~6]{LinckArne2016}.

\section{Cross-Platform Approaches}
The major goal cross-platform approaches try to aim for is only having to develop and maintain one code base for all different platforms. Other important goals are accessing all platform dependent features and to offer the same performance and look and feel a native application does \cite[p.~1]{7479278} \cite[p.~1]{7934674}. To achieve these goals different solutions were developed. These solutions are the following: 
\begin{itemize}
\item using an interpreter to call native APIs at runtime \cite[p.~3]{7479278} \cite[p.~4-5]{LinckArne2016} \cite[p.~5-6]{Hansson_Vidhall_2016}.
\item utilizing a cross-compiler to compile non-native code into a native application \cite[p.~3-4]{7479278} \cite[p.~5]{Hansson_Vidhall_2016}.
\item developing a web application and displaying it via the browser of the mobile device \cite[p.~2]{7479278} \cite[p.~2-3]{LinckArne2016} \cite[p.~4-5]{Hansson_Vidhall_2016}.
\item wrapping a web application into a native webview-component \cite[p.~2-3]{7479278} \cite[p.~3-4]{LinckArne2016} \cite[p.~5]{Hansson_Vidhall_2016}.
\item working with Model driven architecture defined by the Object Management Group to create models that will later on be transformed into native source code \cite[p.~4]{7479278} \cite[p.~3]{7934674}.
\end{itemize}
Every approach besides the ones that transform or compile non-native to native code before runtime suffer from performance issues in comparison with native applications. The worse performance arises, as a result of using an additional layer (browser, webview-component, interpreter) to communicate with the features of the device (e.g. camera) \cite[p.~2,~10]{JohanssonSderberg2018} \cite[p.~5-6]{LinckArne2016} \cite[p.~111]{Keist2016}.
\\[\baselineskip]
While there may be multiple solutions, each approach has its own strengths and weaknesses depending on the situation, therefore none is the best for every possible scenario \cite[p.~110]{Keist2016}. The following chapters describe for each approach how the given approach solves the cross-platform problem, what strengths and weaknesses the given approach has and which frameworks or technologies make use of the given approach.

\subsection{Web Approach}
The solution this approach uses are the browsers of the mobile devices \cite[p.~2]{7934674}. This is possible through the compliance of different browser manufacturers with the HTML standard of the W3C \cite[p.~2]{LinckArne2016}. Therefore by utilizing HTML, CSS, and JavaScript or other web development technologies the developed application can be used on any device that has a browser installed.
\\[\baselineskip]
Because the application can be accessed through the browser nothing has to be installed on the mobile device to use it. In addition, due to web applications are hosted on servers, the browser of the mobile device only needs to view data. Even though having to install nothing is an advantage for the users, the distribution of the application is harder since the mobile application stores cannot be used. Furthermore through connection and network issues the performance of web applications is inferior to native applications. Another problem is that web applications have no way to access hardware features like the camera as a result of the application sandboxing \cite[p.~626]{6420693}. Possible technologies that are based on this approach are Jquery mobile, Sencha touch, and Boostrap \cite[p.~2]{7934674}.

\subsection{Hybrid Approach}
\cite{7479278}, \cite{7934674}

\subsection{Model Driven Approach}
The basis for this approach is provided by the Model Driven Architecture of the Object Management Group. The developer creates models with the functionality required for the application. These models are referred to as Computational Independent Models. As shown in \ref{Fig1}, these undergo several transformations to yield platform specific source code at the end. At first the created models are transformed into Platform Independent Models (PIM). These are then converted to Platform Specific Models (PSM) through the model-to-model transformation. PSMs are, as the name implies, different for each platform and are ultimately transformed into platform-specific source code through a model-to-text transformation. This model-to-text transformation consists most of the time on a template-based approach \cite[p.~4]{7479278} \cite[p.~3]{7934674}.
\begin{figure}[!htbp]
\centering
\includegraphics[width=0.5\linewidth]{PICs/MDA.png}
\caption{Transformation from model to source code for Android and iOS \cite[p.~4]{7479278} \cite[p.~3]{7934674}}\label{Fig1}
\end{figure}
\\[\baselineskip]
The problem with this approach is that there is no mature framework based on it. As a result, the full potential of this approach has not yet been demonstrated. Therefore papers regarding the advantages and disadvantages of this approach are rare. However, a prototype framework called MD2 \cite{MD2} already exists, but since it is still relatively young and no official version was released till now, it is still limited \cite[p.~3-4]{7934674}.

\subsection{Cross-compiled Approach}
\cite{7479278}, \cite{7934674} Xamarin, Flutter,
\begin{figure}[!htbp]
\centering
\includegraphics[width=0.5\linewidth]{PICs/Cross-Compiled.png}
\caption{Cross-compiled approach for Android and iOS \cite[p.~3]{7479278}}\label{Fig2}
\end{figure}

\subsection{Interpreted Approach}
React Native \cite[p.~3]{JohanssonSderberg2018}, \cite{7479278}
\begin{figure}[!htbp]
\centering
\includegraphics[width=0.5\linewidth]{PICs/Interpreted.png}
\caption{Interpreted approach for Android and iOS \cite[p.~3]{7479278}}\label{Fig3}
\end{figure}


\chapter{Selected Frameworks}
Because this paper focuses on presenting a comparison between technologies using either the cross-compiled or interpreted approach, two frameworks had to be chosen. The selected ones are Xamarin and React Native but theoretically there could have been more frameworks selected. However, working with more than one technology per approach would lead to comparing the implementations of the same approach with each other which is not the aim of this thesis. 
\\[\baselineskip]
Though selecting exactly two frameworks does not completely eliminate the possibility of only comparing two implementations with each other that just used a different approach as a solution. Furthermore if one of these frameworks had a bad implementation it could yield far worse results another framework using the same approach could achieve. This is why the selected frameworks had to be of high quality, but testing every known cross-platform framework would have exceeded the scope of this thesis, so lots of researching for related work had to be done. 
\\[\baselineskip]
The biggest reasons for selecting Xamarin and React Native were that both support Android and iOS \cite[p.~1]{JohanssonSderberg2018} \cite[p.~12]{ZubaBernhard2017EdPb}, that both still get new features, bug fixes and improvements \cite{XamarinRoadmap} \cite{ReactNativeRoadmap} and a good amount of valuable studies were available to work with (e.g. \cite{Hansson_Vidhall_2016}, \cite{MartinezLecomte2018}, \cite{GaouarBenamarBendimerad2016}).

\section{Xamarin}
\cite[p.~14-20]{ZubaBernhard2017EdPb} picture how cross compiler works

\section{React Native}
\cite{Hansson_Vidhall_2016} \cite{Danielsson_2016} \cite[p.~21-32]{ZubaBernhard2017EdPb} picture how native bridge works
\cite[p.~14-20]{ZubaBernhard2017EdPb} picture how cross compiler works
\begin{figure}[!htbp]
\centering
\includegraphics[width=0.75\linewidth]{PICs/Bachelor1_NativeBridge.png}
\caption{Native Bridge for Android and iOS \cite{PicReactNativeBridge} \cite[p.~28]{ZubaBernhard2017EdPb}}\label{Fig4}
\end{figure}

\chapter{Evaluation}
% the introduction of this chapter should include why these criteria were chosen
\cite[p.~24]{Danielsson_2016}

\section{Criteria}
% introduction of this chapter should give a little start on each criteria and say that each criteria describes the evaluation of the study like in \cite[p.~24]{Danielsson_2016}

\subsection{Performance}
\cite[p.~25-26]{Danielsson_2016} \cite[p.~30]{Axelsson2016} \cite[p.~29-31]{Hansson_Vidhall_2016}

\subsection{Code Sharing}
\cite[p.~31]{Hansson_Vidhall_2016}

\subsection{Documentation}
The purpose of this criteria was to get an idea of how difficult it is to start developing cross-platform applications with either Xamarin or React Native. However while researching into studies on Xamarin and/or React Native regarding how good or bad the official documentation is, little to no studies were found. The only time authors wrote about the documentation was when they experimented with developing a cross-platform application with either React Native, Xamarin or both. The information contained therein consists mostly of explanations of how the documentation has helped them to solve problems or which information was missing in the official documentation. Some examples of this theses and papers are as follows: developing a React Native application \cite[p.~16-18]{Danielsson_2016}, developing a geolocation and a bluetooth feature, a shared library and creating native bindings for a Xamarin cross-platform application \cite[p.~10-15]{Dickson_2013} and porting a HTML5 web application to a React Native cross-platform application and a Xamarin cross-platform application \cite[p.~33-69]{ZubaBernhard2017EdPb}.
\\[\baselineskip]
Considering this led to changing the purpose of this criteria. Before the aim was to compare the official documentations of Xamarin and React Native with each other and finding out if their documentations lack in some form or another. Now the objective is to summarize experiences developers of React Native and Xamarin cross-platform applications had with using the official documentation of either Xamarin or React Native. 

\subsection{Look and Feel}
\cite[p.~18]{GaouarBenamarBendimerad2016}, react native \cite[p.~25]{Danielsson_2016} \cite[p.~31]{Hansson_Vidhall_2016}


\section{Results}
The following chapters showcase and discuss results of studies and experiments regarding the two cross platform mobile development frameworks Xamarin and React Native. When possible, studies were chosen which underwent experiments with both frameworks. However, lots of other noteworthy studies which only underwent experiments with one of the two frameworks are also featured. In this case the results are not comparable, nevertheless not including them would mean ignoring interesting data on the above mentioned criteria. By incorporating these studies, one gains a better picture of Xamarin and React Native regarding performance, code sharing, documentation and look and feel. The studies, which used both frameworks for their experiments, also provided comparisons and discussions of the results. These comparisons and discussions are also summarized in the following chapters.

\subsection{Performance}
xamarin, react native \cite[p.~30-32]{ZubaBernhard2017EdPb}, xamarin \cite{Armgren_2015} \cite{WillocxVossaertNaessens2015}, react native \cite[p.~67-68]{Axelsson2016} \cite[p.~34-43]{Hansson_Vidhall_2016}
\subsection{Code Sharing}
xamarin, react native \cite[p.~71-72]{ZubaBernhard2017EdPb}, xamarin \cite[p.~185]{MartinezLecomte2017}, react native (results\cite[p.~44]{Hansson_Vidhall_2016}) (discussion\cite[p.~53]{Hansson_Vidhall_2016}),

\subsection{Documentation}
At first, the results of the papers concerning the official documentation of React Native are presented, followed by those of Xamarin. At the end other possible sources for information regarding Xamarin and React Native are listed. 
\\[\baselineskip]
The following list includes the experiences authors of theses, studies or papers had with the documentation of React Native \cite{ReactNativeDoc} while developing cross-platform applications with React Native: 
\begin{itemize}
\item Setting up the development environment for Android and iOS regarding the operating system used on the computer of the developer, connecting to a physical mobile device, setting up a virtual mobile device and creating a new project lead to no problems for William Danielsson when following the steps of the documentations chapter "Getting Started" \cite[p.~18]{Danielsson_2016}.
\item Implementing a feature to capture a photo with the camera of the mobile device poses some problems. One of the problems William Danielsson had to face was that React Native did not support a component for the camera at the time of his experiment. However, Lochlan Wansbrough, a creator of many React Native components, created an open-source camera component. This components repository \cite{ReactNativeCamera} is part of the official React Native community account \cite{ReactNativeCommunity}. The second problem William Danielsson had was that the documentation of this component only existed for iOS and not for Android. He states that the feature was not implemented for Android because it would have exceeded the scope of his experiment \cite[p.~23-24]{Danielsson_2016}.
\item Figuring out how routing and navigating works in an application developed with React Native led to problems when the only source for information is the official documentation of React Native. William Danielsson solved this problem by consulting the following blog \cite{ReactNativeBlog} \cite[p.~51]{Danielsson_2016}.
\end{itemize}
While developing a Xamarin cross-platform application that features sending messages from one device to another through bluetooth and using the official documentation of Xamarin \cite{XamarinDoc} as a source Jared Dickson faced the problem that the documentation for bluetooth on iOS was nonexistent. The Experiment was finished without implementing the feature for iOS. However, at the time he was writing his thesis "Core Bluetooth" was released which makes it possible to use the bluetooth feature on iOS and would have solved his problem, he states \cite[p.~11]{Dickson_2013}.
\\[\baselineskip]
By analyzing a two site long Q\&A Matias Martinez and Sylvain Lecomte tried to find out more about the documentation of Xamarin's error codes. The data for this study was gathered by mining through the questions asked on the Xamarin Forum and on Stack Overflow that contain an error code thrown by the Xamarin framework. On the Xamarin Forum and on Stack Overflow 1121 and 330 questions respectively were found, which include one or more Xamarin error codes. 87.5\% of these questions on the Xamarin Forum and 92.3\% of these questions on Stack Overflow have one or more answers \cite[p.~1,~4]{MartinezLecomte2018}.
\\[\baselineskip]
The official documentations of Xamarin and React Native are not the only ways to get detailed information regarding the specific framework. Most developers also use StackOverflow as a source for explanations and code examples \cite{MartinezLecomte2018}. Also when searching for solutions or information regarding a problem with the native code Xamarin and React Native developers can take a look in the official documentations of Android and iOS \cite[p.~11]{Dickson_2013}. More possible sources for information are React's documentation for React Native developers \cite{ReactDoc} and Xamarin Forum for Xamarin developers \cite[p.~51]{Danielsson_2016} \cite{MartinezLecomte2018}.

\subsection{Look and Feel}
xamarin \cite[p.~21]{GaouarBenamarBendimerad2016}, react native (results\cite[p.~29-31]{Danielsson_2016}) (discussion\cite[p.~45]{Danielsson_2016}) (results\cite[p.~44]{Hansson_Vidhall_2016}) (discussion\cite[p.~53-55]{Hansson_Vidhall_2016})


\chapter{Conclusion \& Future Work}








% Hier beginnen die Verzeichnisse.
\clearpage
\ifthenelse{\equal{\FHTWCitationType}{HARVARD}}{}{\bibliographystyle{gerabbrv}}
\bibliography{Literatur}
\clearpage

% Das Abbildungsverzeichnis
\listoffigures
\clearpage

% Das Tabellenverzeichnis
\listoftables
\clearpage

\phantomsection
\addcontentsline{toc}{chapter}{\listacroname}
\chapter*{\listacroname}
\begin{acronym}[XXXXX]
    \acro{iOS}[iOS]{iPhone Operating System}
    \acro{SDK}[SDK]{Software Development Kit}
    \acro{API}[API]{Application Programming Interface}
    \acro{W3C}[W3C]{World Wide Web Consortium}
    \acro{QnA}[Q\&A]{Questions and Answers}
\end{acronym}

\end{document}}