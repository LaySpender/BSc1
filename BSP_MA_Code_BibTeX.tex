%
% FH Technikum Wien
% !TEX encoding = UTF-8 Unicode
%
% Erstellung von Master- und Bachelorarbeiten an der FH Technikum Wien mit Hilfe von LaTeX und der Klasse TWBOOK
%
% Um ein eigenes Dokument zu erstellen, müssen Sie folgendes ergänzen:
% 1) Mit \documentclass[..] einstellen: Master- oder Bachelorarbeit, Studiengang und Sprache
% 2) Mit \newcommand{\FHTWCitationType}.. Zitierstandard festlegen (wird in der Regel vom Studiengang vorgegeben - bitte erfragen)
% 3) Deckblatt, Kurzfassung, etc. ausfüllen
% 4) und die Arbeit schreiben (die verwendeten Literaturquellen in Literatur.bib eintragen)
%
% Getestet mit TeXstudio mit Zeichenkodierung ISO-8859-1 (=ansinew/latin1) und MikTex unter Windows
% Zu beachten ist, dass die Kodierung der Datei mit der Kodierung des paketes inputenc zusammen passt!
% Die Kodierung der Datei twbook.cls MUSS ANSI betragen!
% Bei der Verwendung von UTF8 muss dnicht nur die Kodierung des Dokuments auf UTF8 gestellt sein, sondern auch die des BibTex-Files!
%
% Bugreports und Feedback bitte per E-Mail an latex@technikum-wien.at
%
% Versionen
% *) V0.7: 9.1.2015, RO: Modeline angepasst und verschoben
% *) V0.6: 10.10.2014, RO: Weitere Anpassung an die UK
% *) V0.5: 8.8.2014, WK: Literaturquellen überarbeitet und angepasst
% *) V0.4: 4.8.2014, WK: Initalversion in SVN eingespielt
%
\documentclass[Bachelor,BIF,english]{twbook}
\usepackage[utf8]{inputenc}
\usepackage[T1]{fontenc}

%
% Bitte in der folgenden Zeile den Zitierstandard festlegen
\newcommand{\FHTWCitationType}{IEEE} % IEEE oder HARVARD möglich - wenn Sie zwischen IEEE und HARVARD wechseln, bitte die temorären Dateien (aux, bbl, ...) löschen
%
\ifthenelse{\equal{\FHTWCitationType}{HARVARD}}{\usepackage{harvard}}{\usepackage{bibgerm}}

% Definition Code-Listings Formatierung:
\usepackage[final]{listings}
\lstset{captionpos=b, numberbychapter=false,caption=\lstname,frame=single, numbers=left, stepnumber=1, numbersep=2pt, xleftmargin=15pt, framexleftmargin=15pt, numberstyle=\tiny, tabsize=3, columns=fixed, basicstyle={\fontfamily{pcr}\selectfont\footnotesize}, keywordstyle=\bfseries, commentstyle={\color[gray]{0.33}\itshape}, stringstyle=\color[gray]{0.25}, breaklines, breakatwhitespace, breakautoindent}
\lstloadlanguages{[ANSI]C, C++, [gnu]make, gnuplot, Matlab}

%Formatieren des Quellcodeverzeichnisses
\makeatletter
% Setzen der Bezeichnungen für das Quellcodeverzeichnis/Abkürzungsverzeichnis in Abhängigkeit von der eingestellten Sprache
\providecommand\listacroname{}
\@ifclasswith{twbook}{english}
{%
    \renewcommand\lstlistingname{Code}
    \renewcommand\lstlistlistingname{List of Code}
    \renewcommand\listacroname{List of Abbreviations}
}{%
    \renewcommand\lstlistingname{Quellcode}
    \renewcommand\lstlistlistingname{Quellcodeverzeichnis}
    \renewcommand\listacroname{Abkürzungsverzeichnis}
}
% Wenn die Option listof=entryprefix gewählt wurde, Definition des Entyprefixes für das Quellcodeverzeichnis. Definition des Macros listoflolentryname analog zu listoflofentryname und listoflotentryname der KOMA-Klasse
\@ifclasswith{scrbook}{listof=entryprefix}
{%
    \newcommand\listoflolentryname\lstlistingname
}{%
}
\makeatother
\newcommand{\listofcode}{\phantomsection\lstlistoflistings}

% Die nachfolgenden Pakete stellen sonst nicht benötigte Features zur Verfügung
\usepackage{blindtext}

%
% Einträge für Deckblatt, Kurzfassung, etc.
%
\title{A Comparative View of Cross-compiled and Interpreted Cross-Platform Approaches for iOS and Android Mobile Application Development}
\author{Dominik Hack}
\studentnumber{1610257044}
\supervisor{Dipl.-Ing. Dr.techn. Thomas Polzer}
\place{Vienna}
\kurzfassung{kurzfassung}
\schlagworte{Schlagwort1, Schlagwort2, Schlagwort3, Schlagwort4}
\outline{abstract}
\keywords{Keyword1, Keyword2, Keyword3, Keyword4}

\begin{document}

%Festlegungen für den HARVARD-Zitierstandard
\ifthenelse{\equal{\FHTWCitationType}{HARVARD}}{
\bibliographystyle{Harvard_FHTW_MR}%Zitierstandard FH Technikum Wien, Studiengang Mechatronik/Robotik, Version 1.2e
\citationstyle{dcu}%Correct citation-style (Harvardand, ";" between citations, "," between author and year)
\citationmode{abbr}%use "et al." with first citation
\iflanguage{ngerman}{
    %Deutsch Neue Rechtschreibung
    \newcommand{\citepic}[1]{(Quelle: \protect\cite{#1})}%Zitat: Bild
    \newcommand{\citefig}[2]{(Quelle: \protect\cite{#1}, S. #2)}%Zitat: Bild aus Dokument
    \newcommand{\citefigm}[2]{(Quelle: modifiziert "ubernommen aus \protect\cite{#1}, S. #2)}%Zitat: modifiziertes Bild aus Dokument
    \newcommand{\citep}{\citeasnoun}%In-Line Zitiat entweder mit \citep{} oder \citeasnoun{}
    \newcommand{\acessedthrough}{Verf{\"u}gbar unter:}%Für URL-Angabe
    \newcommand{\acessedthroughp}{Verf{\"u}gbar bei:}%Für URL-Angabe (Geschützte Datenbank, Zugriff durch FH)
    \newcommand{\acessedat}{Zugang am}%Für URL-Datum-Angabe
    \newcommand{\singlepage}{S.}%Für Seitenangabe (einzelne Seite)
    \newcommand{\multiplepages}{S.}%Für Seitenangabe (mehrere Seiten)
    \newcommand{\chapternr}{K.}%Für Kapitelangabe
    \renewcommand{\harvardand}{\&}%Harvardand in Zitaten
    \newcommand{\abstractonly}{ausschließlich Abstract}
    \newcommand{\edition}{. Auflage}%Angabe der Auflage
}{
\iflanguage{german}{
    %Deutsch
    \newcommand{\citepic}[1]{(Quelle: \protect\cite{#1})}%Zitat: Bild
    \newcommand{\citefig}[2]{(Quelle: \protect\cite{#1}, S. #2)}%Zitat: Bild aus Dokument
    \newcommand{\citefigm}[2]{(Quelle: modifiziert "ubernommen aus \protect\cite{#1}, S. #2)}%Zitat: modifiziertes Bild aus Dokument
    \newcommand{\citep}{\citeasnoun}%In-Line Zitiat entweder mit \citep{} oder \citeasnoun{}
    \newcommand{\acessedthrough}{Verf{\"u}gbar unter:}%Für URL-Angabe
    \newcommand{\acessedthroughp}{Verf{\"u}gbar bei:}%Für URL-Angabe (Geschützte Datenbank, Zugriff durch FH)
    \newcommand{\acessedat}{Zugang am}%Für URL-Datum-Angabe
    \newcommand{\singlepage}{S.}%Für Seitenangabe (einzelne Seite)
    \newcommand{\multiplepages}{S.}%Für Seitenangabe (mehrere Seiten)
    \newcommand{\chapternr}{K.}%Für Kapitelangabe
    \renewcommand{\harvardand}{\&}%Harvardand in Zitaten
    \newcommand{\abstractonly}{ausschließlich Abstract}
    \newcommand{\edition}{. Auflage}%Angabe der Auflage
}{
    %Englisch
    \newcommand{\citepic}[1]{(Source: \protect\cite{#1})}%Zitat: Bild
    \newcommand{\citefig}[2]{(Source: \protect\cite{#1}, p. #2)}%Zitat: Bild aus Dokument
    \newcommand{\citefigm}[2]{(Source: taken with modification from \protect\cite{#1}, p. #2)}%Zitat: modifiziertes Bild aus Dokument
    \newcommand{\citep}{\citeasnoun}%In-Line Zitiat entweder mit \citep{} oder \citeasnoun{}
    \newcommand{\acessedthrough}{Available at:}%Für URL-Angabe
    \newcommand{\acessedthroughp}{Available through:}%Für URL-Angabe (Geschützte Datenbank, Zugriff durch FH)
    \newcommand{\acessedat}{Accessed}%Für URL-Datum-Angabe
    \newcommand{\singlepage}{p.}%Für Seitenangabe (einzelne Seite)
    \newcommand{\multiplepages}{pp.}%Für Seitenangabe (mehrere Seiten)
    \newcommand{\chapternr}{Ch.}%Für Kapitelangabe
    \renewcommand{\harvardand}{\&}%Harvardand in Zitaten
    \newcommand{\abstractonly}{Abstract only}
    \newcommand{\edition}{~edition}%Edition -> note, that you have to write "edition = {2nd},"!
}}}

\maketitle
% .. und hier beginnt die eigentliche Arbeit. Viel Erfolg beim Verfassen!
\chapter{Introduction}
% Motivation (warum arbeitet man überhaupt daran?, allgemeine Einführung in Themengebiete (Einstieg soll sich auf Wissen der Zielgruppe beschränken), State of the art zeigen (Nennung von bereits laufenden bzw. abgeschlossenen Forschungen), Forschungsfragen die man mittels dieser Arbeit beantworten will (hier beginnt roter Faden der Arbeit!), Überblick über die folgenden Kapitel

\chapter{Mobile Development Approaches}
To this day lots of companies if not all recognized that using mobile applications on smartphones or other mobile devices is crucial to have a chance against the competition \cite[p.~1]{7479278}. If a company chooses to offer a mobile application for their customers they do not want to offer the mobile application for only half of their customers. However to reach all customers a company has to develop their mobile application for multiple platforms. Nevertheless without precisely planning the development and maintenance of multiple mobile applications for multiple platforms the company can invest more money into the development than possibly needed \cite[p.~1]{JohanssonSderberg2018} \cite[p.~5,~8]{Steczko2016} \cite[p.~757]{Ciman2014}. This is one of the reasons why choosing the correct mobile development approach is essential for success. To assist in finding the most valuable approach for a project the following points summarize the most used mobile development approaches. 

\section{Native Approach}
Choosing the native development approach implicates developing the same application for each platform the application should be available on. The supported platforms are lots of the time the most commonly used operating systems for mobile devices on the market which to this day are Android (insert company) and iOS (insert company) \cite[p.~5]{Steczko2016}. As a result, two applications have to be developed one for Android by using Java as the programming language and the Android-SDK, and one for iOS by using Swift as the programming language and the iOS-SDK \cite[p.~5]{LinckArne2016} \cite{AppleGettingStarted}. Even though this leads to the requirement of different skill sets and therefore potentially more developers being necessary for the development of the application, it also leads to applications developed by utilizing the official platform specific frameworks and tools recommended by the publishers of the platforms. Through using platform specific frameworks and tools a developer is able to directly communicate with the operating system of the mobile device to get access to features like sensors or cloud services without having to rely on other alternatives most of the cross-platform approaches have to use to communicate with the operating system and get access to the above mentioned features \cite[p.~6]{LinckArne2016}. performance is pretty good. On the contrary two different code bases have to be maintained which further results in more time needed for maintenance or possibly like in the development of the application for each platform developers are necessary.  only one platform with the tools and framework the platform supports \cite[p.~5]{LinckArne2016}, platform specific frameworks support the use of features like sensors or cloud services that are unique for each platform \cite[p.~6]{LinckArne2016}, running directly on the operating system \cite[p.~6]{LinckArne2016},

\section{Cross-Platform Approaches}
One of the goals cross-platform approaches try to achieve is only having to develop and maintain one code basis for all different platforms \cite[p.~1]{7479278}. While there are multiple solutions each approach has their own strengths and weaknesses depending on the situation therefor none is the best for every possible scenario \cite{JohanssonSderberg2018}. The following chapters describe for each approach how the given approach solves the cross-platform problem, which frameworks or technologies support the given approach and what strengths and weaknesses the given approach has.

\subsection{Web Approach}
based on browsers for mobile devices \cite{7934674}, compliance of different browser manufacturers with the HTML standard of the W3C \cite{JohanssonSderberg2018}, uses HTML, CSS, and JavaScript to implement \cite{7934674}, browser as runtime environment \cite{7934674}, Solutions: Jquery mobile, Sencha touch, and Boostrap \cite{7934674}

\subsection{Hybrid Approach}

\subsection{Interpreted Approach}
React Native

\subsection{Cross-compiled Approach}
Xamarin, Flutter,

\subsection{MDA Approach}


\chapter{Selected Frameworks}
% write as an introduction why react native and xamarin 

\section{React Native}

\section{Xamarin}


\chapter{Evaluation}
% the introduction of this chapter should include why these criteria were chosen

\section{Criteria}

\subsection{Performance}

\subsection{Code Sharing}

\subsection{Documentation and Community}

\subsection{Look and Feel}

\section{Results}

\subsection{Performance}

\subsection{Code Sharing}

\subsection{Documentation and Community}

\subsection{Look and Feel}


\chapter{Conclusion \& Future Work}


% Hier beginnen die Verzeichnisse.
\clearpage
\ifthenelse{\equal{\FHTWCitationType}{HARVARD}}{}{\bibliographystyle{gerabbrv}}
\bibliography{Literatur}
\clearpage

% Das Abbildungsverzeichnis
\listoffigures
\clearpage

% Das Tabellenverzeichnis
\listoftables
\clearpage

\phantomsection
\addcontentsline{toc}{chapter}{\listacroname}
\chapter*{\listacroname}
\begin{acronym}[XXXXX]
    \acro{ABC}[ABC]{Alphabet}
    \acro{WWW}[WWW]{world wide web}
    \acro{ROFL}[ROFL]{Rolling on floor laughing}
\end{acronym}

\end{document}