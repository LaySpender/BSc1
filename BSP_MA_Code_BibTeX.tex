%
% FH Technikum Wien
% !TEX encoding = UTF-8 Unicode
%
% Erstellung von Master- und Bachelorarbeiten an der FH Technikum Wien mit Hilfe von LaTeX und der Klasse TWBOOK
%
% Um ein eigenes Dokument zu erstellen, müssen Sie folgendes ergänzen:
% 1) Mit \documentclass[..] einstellen: Master- oder Bachelorarbeit, Studiengang und Sprache
% 2) Mit \newcommand{\FHTWCitationType}.. Zitierstandard festlegen (wird in der Regel vom Studiengang vorgegeben - bitte erfragen)
% 3) Deckblatt, Kurzfassung, etc. ausfüllen
% 4) und die Arbeit schreiben (die verwendeten Literaturquellen in Literatur.bib eintragen)
%
% Getestet mit TeXstudio mit Zeichenkodierung ISO-8859-1 (=ansinew/latin1) und MikTex unter Windows
% Zu beachten ist, dass die Kodierung der Datei mit der Kodierung des paketes inputenc zusammen passt!
% Die Kodierung der Datei twbook.cls MUSS ANSI betragen!
% Bei der Verwendung von UTF8 muss dnicht nur die Kodierung des Dokuments auf UTF8 gestellt sein, sondern auch die des BibTex-Files!
%
% Bugreports und Feedback bitte per E-Mail an latex@technikum-wien.at
%
% Versionen
% *) V0.7: 9.1.2015, RO: Modeline angepasst und verschoben
% *) V0.6: 10.10.2014, RO: Weitere Anpassung an die UK
% *) V0.5: 8.8.2014, WK: Literaturquellen überarbeitet und angepasst
% *) V0.4: 4.8.2014, WK: Initalversion in SVN eingespielt
%
\documentclass[Bachelor,BIF,english]{twbook}
\usepackage[utf8]{inputenc}
\usepackage[T1]{fontenc}

%
% Bitte in der folgenden Zeile den Zitierstandard festlegen
\newcommand{\FHTWCitationType}{IEEE} % IEEE oder HARVARD möglich - wenn Sie zwischen IEEE und HARVARD wechseln, bitte die temorären Dateien (aux, bbl, ...) löschen
%
\ifthenelse{\equal{\FHTWCitationType}{HARVARD}}{\usepackage{harvard}}{\usepackage{bibgerm}}

% Definition Code-Listings Formatierung:
\usepackage[final]{listings}
\lstset{captionpos=b, numberbychapter=false,caption=\lstname,frame=single, numbers=left, stepnumber=1, numbersep=2pt, xleftmargin=15pt, framexleftmargin=15pt, numberstyle=\tiny, tabsize=3, columns=fixed, basicstyle={\fontfamily{pcr}\selectfont\footnotesize}, keywordstyle=\bfseries, commentstyle={\color[gray]{0.33}\itshape}, stringstyle=\color[gray]{0.25}, breaklines, breakatwhitespace, breakautoindent}
\lstloadlanguages{[ANSI]C, C++, [gnu]make, gnuplot, Matlab}

%Formatieren des Quellcodeverzeichnisses
\makeatletter
% Setzen der Bezeichnungen für das Quellcodeverzeichnis/Abkürzungsverzeichnis in Abhängigkeit von der eingestellten Sprache
\providecommand\listacroname{}
\@ifclasswith{twbook}{english}
{%
    \renewcommand\lstlistingname{Code}
    \renewcommand\lstlistlistingname{List of Code}
    \renewcommand\listacroname{List of Abbreviations}
}{%
    \renewcommand\lstlistingname{Quellcode}
    \renewcommand\lstlistlistingname{Quellcodeverzeichnis}
    \renewcommand\listacroname{Abkürzungsverzeichnis}
}
% Wenn die Option listof=entryprefix gewählt wurde, Definition des Entyprefixes für das Quellcodeverzeichnis. Definition des Macros listoflolentryname analog zu listoflofentryname und listoflotentryname der KOMA-Klasse
\@ifclasswith{scrbook}{listof=entryprefix}
{%
    \newcommand\listoflolentryname\lstlistingname
}{%
}
\makeatother
\newcommand{\listofcode}{\phantomsection\lstlistoflistings}

% Die nachfolgenden Pakete stellen sonst nicht benötigte Features zur Verfügung
\usepackage{blindtext}
\usepackage{parskip}

%
% Einträge für Deckblatt, Kurzfassung, etc.
%
\title{A Comparative View of Cross-compiled and Interpreted Cross-Platform Approaches for iOS and Android Mobile Application Development}
\author{Dominik Hack}
\studentnumber{1610257044}
\supervisor{Dipl.-Ing. Dr.techn. Thomas Polzer}
\place{Vienna}
\kurzfassung{kurzfassung}
\schlagworte{Schlagwort1, Schlagwort2, Schlagwort3, Schlagwort4}
\outline{abstract}
\keywords{Keyword1, Keyword2, Keyword3, Keyword4}

\begin{document}

%Festlegungen für den HARVARD-Zitierstandard
\ifthenelse{\equal{\FHTWCitationType}{HARVARD}}{
\bibliographystyle{Harvard_FHTW_MR}%Zitierstandard FH Technikum Wien, Studiengang Mechatronik/Robotik, Version 1.2e
\citationstyle{dcu}%Correct citation-style (Harvardand, ";" between citations, "," between author and year)
\citationmode{abbr}%use "et al." with first citation
\iflanguage{ngerman}{
    %Deutsch Neue Rechtschreibung
    \newcommand{\citepic}[1]{(Quelle: \protect\cite{#1})}%Zitat: Bild
    \newcommand{\citefig}[2]{(Quelle: \protect\cite{#1}, S. #2)}%Zitat: Bild aus Dokument
    \newcommand{\citefigm}[2]{(Quelle: modifiziert "ubernommen aus \protect\cite{#1}, S. #2)}%Zitat: modifiziertes Bild aus Dokument
    \newcommand{\citep}{\citeasnoun}%In-Line Zitiat entweder mit \citep{} oder \citeasnoun{}
    \newcommand{\acessedthrough}{Verf{\"u}gbar unter:}%Für URL-Angabe
    \newcommand{\acessedthroughp}{Verf{\"u}gbar bei:}%Für URL-Angabe (Geschützte Datenbank, Zugriff durch FH)
    \newcommand{\acessedat}{Zugang am}%Für URL-Datum-Angabe
    \newcommand{\singlepage}{S.}%Für Seitenangabe (einzelne Seite)
    \newcommand{\multiplepages}{S.}%Für Seitenangabe (mehrere Seiten)
    \newcommand{\chapternr}{K.}%Für Kapitelangabe
    \renewcommand{\harvardand}{\&}%Harvardand in Zitaten
    \newcommand{\abstractonly}{ausschließlich Abstract}
    \newcommand{\edition}{. Auflage}%Angabe der Auflage
}{
\iflanguage{german}{
    %Deutsch
    \newcommand{\citepic}[1]{(Quelle: \protect\cite{#1})}%Zitat: Bild
    \newcommand{\citefig}[2]{(Quelle: \protect\cite{#1}, S. #2)}%Zitat: Bild aus Dokument
    \newcommand{\citefigm}[2]{(Quelle: modifiziert "ubernommen aus \protect\cite{#1}, S. #2)}%Zitat: modifiziertes Bild aus Dokument
    \newcommand{\citep}{\citeasnoun}%In-Line Zitiat entweder mit \citep{} oder \citeasnoun{}
    \newcommand{\acessedthrough}{Verf{\"u}gbar unter:}%Für URL-Angabe
    \newcommand{\acessedthroughp}{Verf{\"u}gbar bei:}%Für URL-Angabe (Geschützte Datenbank, Zugriff durch FH)
    \newcommand{\acessedat}{Zugang am}%Für URL-Datum-Angabe
    \newcommand{\singlepage}{S.}%Für Seitenangabe (einzelne Seite)
    \newcommand{\multiplepages}{S.}%Für Seitenangabe (mehrere Seiten)
    \newcommand{\chapternr}{K.}%Für Kapitelangabe
    \renewcommand{\harvardand}{\&}%Harvardand in Zitaten
    \newcommand{\abstractonly}{ausschließlich Abstract}
    \newcommand{\edition}{. Auflage}%Angabe der Auflage
}{
    %Englisch
    \newcommand{\citepic}[1]{(Source: \protect\cite{#1})}%Zitat: Bild
    \newcommand{\citefig}[2]{(Source: \protect\cite{#1}, p. #2)}%Zitat: Bild aus Dokument
    \newcommand{\citefigm}[2]{(Source: taken with modification from \protect\cite{#1}, p. #2)}%Zitat: modifiziertes Bild aus Dokument
    \newcommand{\citep}{\citeasnoun}%In-Line Zitiat entweder mit \citep{} oder \citeasnoun{}
    \newcommand{\acessedthrough}{Available at:}%Für URL-Angabe
    \newcommand{\acessedthroughp}{Available through:}%Für URL-Angabe (Geschützte Datenbank, Zugriff durch FH)
    \newcommand{\acessedat}{Accessed}%Für URL-Datum-Angabe
    \newcommand{\singlepage}{p.}%Für Seitenangabe (einzelne Seite)
    \newcommand{\multiplepages}{pp.}%Für Seitenangabe (mehrere Seiten)
    \newcommand{\chapternr}{Ch.}%Für Kapitelangabe
    \renewcommand{\harvardand}{\&}%Harvardand in Zitaten
    \newcommand{\abstractonly}{Abstract only}
    \newcommand{\edition}{~edition}%Edition -> note, that you have to write "edition = {2nd},"!
}}}

\maketitle
% .. und hier beginnt die eigentliche Arbeit. Viel Erfolg beim Verfassen!
\chapter{Introduction}
% Motivation (warum arbeitet man überhaupt daran?, allgemeine Einführung in Themengebiete (Einstieg soll sich auf Wissen der Zielgruppe beschränken), State of the art zeigen (Nennung von bereits laufenden bzw. abgeschlossenen Forschungen), Forschungsfragen die man mittels dieser Arbeit beantworten will (hier beginnt roter Faden der Arbeit!), Überblick über die folgenden Kapitel

\chapter{Mobile Development Approaches}
To this day lots of companies if not all recognized that using mobile applications on smartphones or other mobile devices is crucial to have a chance against the competition \cite[p.~1]{7479278}. If a company chooses to offer a mobile application for their customers they do not want to offer the mobile application for only half of their customers. However to reach all customers a company has to develop their mobile application for multiple platforms. 
\\[\baselineskip]
Nevertheless without precisely planning the development and maintenance of multiple mobile applications for multiple platforms the company can invest more money into the development than possibly needed \cite[p.~1]{JohanssonSderberg2018} \cite[p.~5,~8]{Steczko2016} \cite[p.~757]{Ciman2014}. This is one of the reasons why choosing the correct mobile development approach is essential for success. To assist in finding the most valuable approach for a project the following points summarize the most used mobile development approaches. 

\section{Native Approach}
Choosing the native development approach implicates developing the same application for each platform the application should be available on. The supported platforms are lots of the time the most commonly used operating systems for mobile devices on the market which to this day are Android (insert company) and iOS (insert company) \cite[p.~5]{Steczko2016}. As a result, two applications have to be developed one for Android by using Java as the programming language and the Android-SDK, and one for iOS by using Swift as the programming language and the iOS-SDK \cite[p.~5]{LinckArne2016} \cite{AppleGettingStarted}. Even though this leads to the requirement of different skill sets and therefore potentially more developers being necessary for the development of the application, it also leads to applications developed by utilizing the official platform specific frameworks and tools recommended by the publishers of the platforms. 
\\[\baselineskip]
Through using platform specific frameworks and tools a developer is able to directly communicate with the operating system of the mobile device to get access to features like sensors or cloud services without having to rely on other alternatives cross-platform applications have to use to communicate with the operating system and get access to the above mentioned features \cite[p.~6]{LinckArne2016}. These alternatives are as follows: using an interpreter to call native APIs at runtime, utilizing a cross-compiler to compile non-native code into a native application, developing a web application and displaying it via the browser of the mobile device, using a webview-component for a web application or using Model driven architecture defined by the Object Management Group to create models that will later on be transformed into native source code \cite[p.~4--6]{Hansson_Vidhall_2016} \cite[p.~3--5]{LinckArne2016}. Every alternative besides the ones that transform non-native code to native applications or native source code before runtime suffer from performance issues in comparison with native applications \cite[p.~2,~10]{JohanssonSderberg2018} \cite[p.~6]{LinckArne2016} \cite[p.~111]{Keist2016}. 
\\[\baselineskip]
On the contrary each application has a different code base thus each code base has to be maintained. This either means that one developer with knowledge of developing on all platforms needs to work on maintenance for all applications or like in the development of the applications a developer with special knowledge regarding one platform works on the maintenance of this particular platform's application \cite[p.~6]{LinckArne2016}.

\section{Cross-Platform Approaches}
The major goal cross-platform approaches try to achieve is only having to develop and maintain one code base for all different platforms. Other important goals are accessing all platform dependent features, remaining the same performance and look and feel a native application has \cite[p.~1]{7479278} \cite[p.~1]{7934674}. While there are multiple solutions each approach has their own strengths and weaknesses depending on the situation therefor none is the best for every possible scenario \cite[p.~110]{Keist2016}. The following chapters describe for each approach how the given approach solves the cross-platform problem, what strengths and weaknesses the given approach has and which frameworks or technologies make use of the given approach.

\subsection{Web Approach}
based on browsers for mobile devices \cite[p.~2]{7934674}, compliance of different browser manufacturers with the HTML standard of the W3C \cite[p.~2]{LinckArne2016}, uses HTML, CSS, and JavaScript to implement \cite[p.~2]{7934674}, browser as runtime environment \cite{7934674}, Solutions: Jquery mobile, Sencha touch, and Boostrap \cite{7934674}

\subsection{Hybrid Approach}

\subsection{Interpreted Approach}
React Native

\subsection{Cross-compiled Approach}
Xamarin, Flutter,

\subsection{Model Driven Approach}


\chapter{Selected Frameworks}
% write as an introduction why react native and xamarin 

\section{React Native}
\cite{Hansson_Vidhall_2016} \cite{Danielsson_2016} \cite[p.~21-32]{ZubaBernhard2017EdPb}

\section{Xamarin}
\cite[p.~14-20]{ZubaBernhard2017EdPb}


\chapter{Evaluation}
% the introduction of this chapter should include why these criteria were chosen
\cite[p.~24]{Danielsson_2016}

\section{Criteria}
% introduction of this chapter should give a little start on each criteria and say that each criteria describes the evaluation of the study like in \cite[p.~24]{Danielsson_2016}

\subsection{Performance}
\cite[p.~25-26]{Danielsson_2016} \cite[p.~30]{Axelsson2016} \cite[p.~29-31]{Hansson_Vidhall_2016}

\subsection{Code Sharing}
\cite[p.~31]{Hansson_Vidhall_2016}

\subsection{Documentation}
The purpose of this criteria was to get an idea how difficult it is to start developing or develop cross-platform applications with Xamarin or React Native. However while researching into studies on Xamarin and/or React Native regarding how good or bad the official documentation is, little to no studies were found. The only time authors wrote about the documentation was when they experimented with developing a cross-platform application with either React Native, Xamarin or both. The information contained therein consists mostly of explanations of how the documentation has helped them to solve problems or which information was missing in the official documentation. Some examples of this theses and papers are as follows: developing a React Native application \cite[p.~16-18]{Danielsson_2016}, developing a geolocation and a bluetooth feature, a shared library and creating native bindings for a Xamarin cross-platform application \cite[p.~10-15]{Dickson_2013} and porting a HTML5 web application to a React Native cross-platform application and a Xamarin cross-platform application \cite[p.~33-69]{ZubaBernhard2017EdPb}. Considering this led to changing the purpose of this criteria from comparing the documentation of Xamarin and React Native with each other and finding out if the documentation of Xamarin or React Native lacks in some form or another to summarizing experiences developers of React Native and Xamarin cross-platform applications had with using the official documentation of either Xamarin or React Native. 

\subsection{Look and Feel}
\cite[p.~18]{GaouarBenamarBendimerad2016}, react native \cite[p.~25]{Danielsson_2016} \cite[p.~31]{Hansson_Vidhall_2016}


\section{Results}
The following chapters showcase and discuss results of studies and experiments regarding the two cross platform mobile development frameworks Xamarin and React Native. When possible studies were chosen which underwent experiments with both frameworks. However lots of other noteworthy studies which only underwent experiments with one of the two frameworks are also featured. In this case the results are not comparable nevertheless they still showcase results regarding the above mentioned criteria and further help to get a better picture in regard to both frameworks. The studies that did use both frameworks also feature a comparison of their results.

\subsection{Performance}
xamarin, react native \cite[p.~30-32]{ZubaBernhard2017EdPb}, xamarin \cite{WillocxVossaertNaessens2015}, react native \cite[p.~67-68]{Axelsson2016} \cite[p.~34-43]{Hansson_Vidhall_2016}

\subsection{Code Sharing}
xamarin, react native \cite[p.~71-72]{ZubaBernhard2017EdPb}, xamarin \cite[p.~185]{MartinezLecomte2017}, react native (results\cite[p.~44]{Hansson_Vidhall_2016}) (discussion\cite[p.~53]{Hansson_Vidhall_2016}),

\subsection{Documentation}
The official documentations of Xamarin and React Native are not the only ways to get detailed information regarding the specific framework. Most developers also use StackOverflow as a source for explanations and code examples. Also when searching for solutions or information regarding a problem with the native code Xamarin and React Native developers can take a look in the official documentations of Android and iOS \cite[p.~11]{Dickson_2013}. Furthermore React Native developers do not only have React Native's documentation but also React's documentation.
xamarin,
The following list includes the experiences authors of theses, studies or papers had with the official documentation of React Native \cite{ReactNativeDoc} while developing cross-platform applications with React Native: 
\begin{itemize}
\item Setting up the development environment for Android and iOS regarding the operating system used on the computer of the developer, connecting to a physical mobile device, setting up a virtual mobile device and creating a new project lead to no problems for William Danielsson when following the steps of the documentations chapter "Getting Started" \cite[p.~18]{Danielsson_2016}.
\item \cite[p.~23-24]{Danielsson_2016}.
\item Figuring out how routing and navigating works in an application developed with React Native led to problems when only reading the documentation. William Danielsson solved this problem by reading this blog \cite{ReactNativeBlog} \cite[p.~51]{Danielsson_2016}.
\end{itemize}

\subsection{Look and Feel}
xamarin \cite[p.~21]{GaouarBenamarBendimerad2016}, react native (results\cite[p.~29-31]{Danielsson_2016}) (discussion\cite[p.~45]{Danielsson_2016}) (results\cite[p.~44]{Hansson_Vidhall_2016}) (discussion\cite[p.~53-55]{Hansson_Vidhall_2016})


\chapter{Conclusion \& Future Work}








% Hier beginnen die Verzeichnisse.
\clearpage
\ifthenelse{\equal{\FHTWCitationType}{HARVARD}}{}{\bibliographystyle{gerabbrv}}
\bibliography{Literatur}
\clearpage

% Das Abbildungsverzeichnis
\listoffigures
\clearpage

% Das Tabellenverzeichnis
\listoftables
\clearpage

\phantomsection
\addcontentsline{toc}{chapter}{\listacroname}
\chapter*{\listacroname}
\begin{acronym}[XXXXX]
    \acro{iOS}[iOS]{iPhone Operating System}
    \acro{SDK}[SDK]{Software Development Kit}
    \acro{W3C}[W3c]{World Wide Web Consortium}
\end{acronym}

\end{document}